%----------------------------------------------------------------------------------------
%	PACKAGES AND OTHER DOCUMENT CONFIGURATIONS
%----------------------------------------------------------------------------------------

\documentclass{resume} % Use the custom resume.cls style

\usepackage[left=0.75in,top=0.6in,right=0.75in,bottom=0.6in]{geometry} % Document margins
\usepackage[utf8]{inputenc}
\usepackage[T2A]{fontenc}
\usepackage[russian]{babel}
\usepackage{hyperref}
\newcommand{\hreftext}[1]{\underline{\textit{#1}}}

\name{Денис Кутуков}
\address{\href{mailto:mail@dskut.ru}{\hreftext{mail@dskut.ru}} \\ \href{http://dskut.ru}{\hreftext{http://dskut.ru}} \\ Москва}

\begin{document}

%----------------------------------------------------------------------------------------
%	EDUCATION SECTION
%----------------------------------------------------------------------------------------

\begin{rSection}{Образование}

{\bf МГТУ им. Н.Э.Баумана} \hfill {\em 2005~--- 2011} \\
Фак-т <<Информатика и системы управления>> \\ 
Каф. <<Программное обеспечение ЭВМ и информационные технологии>>, ИУ-7 \\
Магистр техники и технологии \\
Тема дипломного проекта: <<Разработка метода онлайн-кластеризации новостного потока>> \\
Средний балл в дипломе: 4.6

\end{rSection}

%----------------------------------------------------------------------------------------
%	WORK EXPERIENCE SECTION
%----------------------------------------------------------------------------------------

\begin{rSection}{Опыт работы}

\begin{rSubsection}{Яндекс}{Июль 2011~--- наст. вр.}{Разработчик}{Москва}
Работа над сервисом Яндекс.Новости, \href{http://news.yandex.ru}{\hreftext{http://news.yandex.ru}}
\item Улучшение алгоритмов аннотирования и кластеризации текстовых новостных сообщений, как по качеству результата, так и по прозводительности.
\item Использование дополнительных доступных данных и словарей, внедрение машинного обучения для повышения точности и полноты алгоритмов.
\item Внедрение метрик качества формирования новостных сюжетов.
\item Улучшение и рефакторинг архитектуры сервиса, поддержка унаследованного кода.
\item Разработка вспомогательных скриптов и инструментов для анализа работы сервиса.
\item Внедрение практики написания юнит-тестов в команде; покрытие тестами кода, за который я отвечаю; разработка модуля непрерывной интеграции для мониторинга метрик качества алгоритмов.
\item Технологии: С++ (в основном), perl, python, XML/XSL/JSON/ProtoBuf, bash/awk, FreeBSD/Ubuntu, Redis/BerkeleyDB.
\end{rSubsection}

%------------------------------------------------

\begin{rSubsection}{Devoteam Teligent}{Апрель 2011~--- Июнь 2011}{Инженер-программист}{Москва}
\item Проектирование и разработка распределенного серверного приложения для аудио- и видеоконференций 
с использованием внутренней платформы Teligent P90.
\item Реализация обработчиков DTMF-сигналов и воспроизведения голосовых сообщений IVR. 
\item Написание юнит-тестов (CppUnit), интеграционных тестов (python) и утилиты 
для нагрузочного тестирования (С++).
\item Технологии: С++, python, bash, RHEL, протоколы SIP/SDP/RTP.
\end{rSubsection}

%------------------------------------------------

\begin{rSubsection}{РАА СпецТехника}{Февраль 2010~--- Февраль 2011}{Инженер-программист}{Москва}
\item Разработка программно-аппаратных комплексов для практического и теоретического обучения военных
летчиков.
\item Проектирование и реализация модулей для управления ходом тренировки с рабочего места инструктора 
и оценки действий обучаемого.
\item Интеграция подсистемы логгирования в приложении рабочего места инструктора.
\item Поддержка и доработка приложения для теоретического обучения и централизованного контроля знаний 
обучаемых (распределенное приложение для использования в компьютерном классе).
\item Технологии: C\# + managed C++, .NET 3.5, WPF, ADO, Linq, MS SQL Server.
\end{rSubsection}

%------------------------------------------------

\begin{rSubsection}{API}{Сентябрь 2009~--- Январь 2010}{Программист}{Москва}
\item Участие в создании проекта-прототипа системы мониторинга и анализа СМИ по регионам РФ.
\item Разработка отдельных компонент интерфейса.
\item Написание хранимых процедур и модуля взаимодействия с БД.
\item Технологии: C\#, .NET 3.5, WPF, MS SQL Server.
\end{rSubsection}

\end{rSection}

%----------------------------------------------------------------------------------------
%	TECHNICAL STRENGTHS SECTION
%----------------------------------------------------------------------------------------

\begin{rSection}{Технические навыки}

%\begin{tabular}{ @{} >{\bfseries}l @{\hspace{6ex}} l }
\textbf{Языки программирования:} \\
C, C++ (STL, boost), python2.*, perl, bash, C\# (.NET), JavaSE, Erlang, Scala \\
\textbf{Базы данных:} \\
MySQL, MS SQL Server, Redis \\
\textbf{IDE:} \\
QtCreator, Eclipse, MS Visual Studio \\
\textbf{Инструменты:} \\
svn, git, JIRA, Redmine, vim, gdb, valgrind, oprofile, make, cmake, tcpdump/wireshark \\
\textbf{Операционные системы:} \\
UNIX (FreeBSD, Debian/Ubuntu, Arch Linux), Windows \\
\textbf{Шаблоны проектирования:} \\
Go4 (Bridge/Factory/Observer), J2EE(DAO/CompositeEntity), MVC \\
\textbf{Прочее:} \\
Опыт многопоточного программирования (разделяемая память, модель акторов); \\
твердое знание теории (базовые структуры данных, сложность алгоритмов, 
машинное обучение, искусственый интеллект); \\
опыт написания полезных юнит-тестов. \\
%\end{tabular}

\end{rSection}

%----------------------------------------------------------------------------------------

\begin{rSection}{Иностранные языки}
\begin{tabular}{ @{} >{\bfseries}l @{\hspace{6ex}} l }
Английский & Advanced (свободное владение) \\
Немецкий & C1 (высокий уровень) \\
Русский & родной 
\end{tabular}
\end{rSection}

%----------------------------------------------------------------------------------------

\begin{rSection}{Курсы и сертификаты}
\begin{itemize}
\item Brainbench C++ Master (\href{http://www.brainbench.com/content/transcript/topicdetail.do?testid=12320176}{\hreftext{Публичный транскрипт}})

\item Deutsch als Fremdsprache (TestDAF)

\item MOOC-курсы на \href{http://coursera.org}{\hreftext{Coursera}}: \href{http://www.coursera.org/user/i/e2d8625ac8ed6c7665b8ad68b990e47b}{\hreftext{мой профиль}} c оконченными курсами, а также \href{http://ml-class.org}{\hreftext{Machine Learning}}, \href{http://ai-class.org}{\hreftext{Artificial Intelligence}}, \href{http://db-class.org}{\hreftext{Databases}} 

\end{itemize}
\end{rSection}

%----------------------------------------------------------------------------------------

\begin{rSection}{Хобби}
Спорт (бокс, фитнес), литература.
\end{rSection}

%----------------------------------------------------------------------------------------

\end{document}
